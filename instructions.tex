\documentclass[paper=81mm:81mm]{scrartcl}
\usepackage[ngerman]{babel}

% PDF links
\usepackage{hyperref}

% amsmath with improvements
\usepackage{mathtools}

% use GoMono as typewriter font
\usepackage[
  % otherwise bold is not available ("undefined font shape" warnings)
  type1,
  % scale to match main font size
  scale=0.88,
]{GoMono}

% need T1 font encoding for Charter,
% otherwise there will be "undefined font shape" warnings
\usepackage[T1]{fontenc}

% silence warnings
\usepackage{silence}

% use Bitstream Charter as main font,
% suppress warning due to https://github.com/latex3/latex2e/issues/299
\WarningFilter{latex}{Font shape declaration has incorrect series value `mc'.}
\usepackage[bitstream-charter]{mathdesign}

% specify fonts to use
\usepackage{fontspec}

% symbols and names of CC licenses
\usepackage[type=CC,modifier=by-sa,version=4.0]{doclicense}

% control enumerate and itemize
\usepackage{enumitem}

% micro-typographic adjustments
\usepackage{microtype}

% colors
\usepackage{xcolor}

% graphics
\usepackage{graphicx}

% drawings
\usepackage{tikz}

% contours
\usepackage[outline]{contour}

% output stuff after the current page
\usepackage{afterpage}

\usetikzlibrary{
  % text along circle
  decorations.text,
}

% set thickness of contours
\contourlength{0.01em}

% set Charter as main font (otherwise, sz ligature will be typeset as "SS")
\setmainfont{XCharter}

% don't use sans-serif font for headings
\setkomafont{disposition}{\normalcolor\bfseries}

% define colors (mix between MATLAB and matplotlib colors)
\definecolor{C0}{rgb}{0.000,0.447,0.741}
\definecolor{C1}{rgb}{0.850,0.325,0.098}

% set up hyperref
\hypersetup{
  pdfauthor={Julian Valentin},
  pdfcreator={LaTeX, KOMA-Script, hyperref},
  % underline links instead of putting a framed box around them
  pdfborderstyle={/S/U/W 1},
  % set link colors
  citebordercolor=C1,
  filebordercolor=C1,
  linkbordercolor=C1,
  menubordercolor=C1,
  runbordercolor=C1,
  urlbordercolor=C0,
  % prepend bookmarks with section number
  bookmarksnumbered,
}

% number sets
\newcommand*{\nat}{\mathbb{N}}
\newcommand*{\integer}{\mathbb{Z}}

% notation
\newcommand*{\field}[1]{\integer/n\integer}
\newcommand*{\powerset}[1]{\mathcal{P}(#1)}

% binary relations
\newcommand*{\ceq}{\coloneqq}
\newcommand*{\dcup}{\mathbin{\dot{\cup}}}

% commands
\newcommand*{\term}[1]{\emph{#1}}
\newcommand*{\theda}[1]{%
  \texorpdfstring{$\vartheta$}{ϑ}~--~\textsc{Das~Spiel}%
  \if\relax\detokenize{#1}\relax%
  \else%
    ~--~#1%
  \fi%
}

% macros for including pages
\newcommand*{\pageSize}{8.1}
\newcommand*{\truncationAngle}{30}
\pgfmathsetmacro{\pageRadius}{\pageSize/sqrt(2)}
\pgfmathsetmacro{\truncationPointTopX}{\pageRadius*cos(180-\truncationAngle)}
\pgfmathsetmacro{\truncationPointTopY}{\pageRadius*sin(180-\truncationAngle)}

% include a single right page
\newcommand*{\rightPage}[4]{
  \if\relax\detokenize{#4}\relax
  \else
    \node at (#1,#2) {\includegraphics[page=#4]{#3}};
    \draw (#1-\pageSize/2,#2+\pageSize/2)
        -- (#1+\pageSize/2,#2+\pageSize/2)
        -- (#1+\pageSize/2,#2-\pageSize/2)
        -- (#1-\pageSize/2,#2-\pageSize/2)
        arc[start angle=225,end angle=180+\truncationAngle,radius=\pageRadius]
        -- (#1+\truncationPointTopX,#2+\truncationPointTopY)
        arc[start angle=180-\truncationAngle,end angle=135,radius=\pageRadius];
  \fi
}

\newcommand*{\leftPage}[4]{
  \if\relax\detokenize{#4}\relax
  \else
    \node at (#1,#2) {\includegraphics[page=#4]{#3}};
    \draw[white] (#1+\pageSize/2,#2+\pageSize/2)
        -- (#1-\pageSize/2,#2+\pageSize/2)
        -- (#1-\pageSize/2,#2-\pageSize/2)
        -- (#1+\pageSize/2,#2-\pageSize/2)
        arc[start angle=-45,end angle=-\truncationAngle,radius=\pageRadius]
        -- (#1-\truncationPointTopX,#2+\truncationPointTopY)
        arc[start angle=\truncationAngle,end angle=45,radius=\pageRadius];
  \fi
}

% place six right pages on one sheet of paper
\newcommand*{\nupRightPage}[7]{%
  \vspace*{\fill}%
  %
  \begin{center}%
    \begin{tikzpicture}
      \rightPage{0}{18}{#1}{#2}
      \rightPage{10}{18}{#1}{#3}
      \rightPage{0}{9}{#1}{#4}
      \rightPage{10}{9}{#1}{#5}
      \rightPage{0}{0}{#1}{#6}
      \rightPage{10}{0}{#1}{#7}
    \end{tikzpicture}%
  \end{center}%
  %
  \vspace*{\fill}%
  \pagebreak%
  \par%
}

% place six left pages on one sheet of paper
\newcommand*{\nupLeftPage}[7]{%
  \vspace*{\fill}%
  %
  \begin{center}%
    \begin{tikzpicture}
      \leftPage{0}{18}{#1}{#2}
      \leftPage{10}{18}{#1}{#3}
      \leftPage{0}{9}{#1}{#4}
      \leftPage{10}{9}{#1}{#5}
      \leftPage{0}{0}{#1}{#6}
      \leftPage{10}{0}{#1}{#7}
    \end{tikzpicture}%
  \end{center}%
  %
  \vspace*{\fill}%
  \pagebreak%
  \par%
}


% set KOMA-Script text area
\areaset{77mm}{77mm}

% set PDF title
\hypersetup{pdftitle={\theda{Die Anleitung}}}

% read symbol names from file
\newread\symbolNamesread
\IfFileExists{./symbolNames.txt}{
  \openin\symbolNamesread=symbolNames.txt
}{
  \openin\symbolNamesread=symbolNames.template.txt
}

\newcommand*{\symbolName}{\read\symbolNamesread to \symbolNameTemp\symbolNameTemp}

% read symbols from file
\newcounter{symbolImageNumber}
\newcounter{symbolImageRow}
\newlength{\symbolImageLeft}
\newlength{\symbolImageBottom}
\newlength{\symbolImageRight}
\newlength{\symbolImageTop}

\IfFileExists{./symbols.pdf}{
  \newcommand*{\symbolImagePath}{symbols.pdf}
}{
  \newcommand*{\symbolImagePath}{symbols.template.pdf}
}

\newcommand*{\symbolImage}[2]{%
  \setcounter{symbolImageRow}{\thesymbolImageNumber/6}%
  \setlength{\symbolImageLeft}{100mm*(\thesymbolImageNumber-6*\thesymbolImageRow)+50mm-100mm*\real{#1}/2}%
  \setlength{\symbolImageBottom}{100mm*(5-\thesymbolImageRow)+50mm-100mm*\real{#2}/2}%
  \setlength{\symbolImageRight}{\symbolImageLeft+100mm*\real{#1}}%
  \setlength{\symbolImageTop}{\symbolImageBottom+100mm*\real{#2}}%
  \includegraphics[
    viewport={{\symbolImageLeft} {\symbolImageBottom} {\symbolImageRight} {\symbolImageTop}},
    clip,
    scale=0.3,
  ]{\symbolImagePath}%
  \addtocounter{symbolImageNumber}{1}%
}

\begin{document}

\vspace*{\fill}

\noindent\hspace*{9mm}%
\begin{tikzpicture}
  \node at (0,0) {%
    \fontsize{200pt}{240pt}\selectfont%
    \contour{black}{\textcolor{white}{$\vartheta$}}%
  };
  \node[align=center,anchor=north,inner sep=0mm] at (3,0.5) {%
    \LARGE\textsc{Das}\\[0.5em]%
    \LARGE\textsc{Spiel}%
  };
  \node[align=center,anchor=north,inner sep=0mm] at (3,-1.3) {%
    \LARGE\textsc{Die}\\[0.5em]%
    \LARGE\textsc{Anleitung}%
  };
  \draw (2.6,-0.95) -- (3.4,-0.95);
\end{tikzpicture}

\vspace*{\fill}

{%
  \noindent\small%
  Bilder/Texte: Copyright \textcopyright{} 2021 Julian Valentin,\\
  lizenziert unter \doclicenseIcon{} CC BY-SA 4.0%
}

\pagebreak

\section*{Ziel des Spiels}

\theda{} (ausgesprochen "`Tedda"')
ist eine kleinere Version von "`Dobble"' und "`Spot It!"' mit eigenen Symbolen.
Ziel des Spiels ist es, als Erste/r alle Karten abzulegen,
indem gleiche Symbole erkannt werden.

\section*{Inhalt}

\begin{itemize}
  \item
  Anleitung

  \item
  Hintergrund

  \item
  31 Karten
\end{itemize}

\section*{Spielablauf}

\theda{} kann mit insgesamt zwei bis fünf Spieler*innen gespielt werden.
Jede/r Spieler*in hat Karten auf der Hand.
Zusätzlich gibt es einen Stapel in der Tischmitte.

\begin{enumerate}
  \item
  Eine beliebige Person fungiert als Kartengeber*in und mischt die 31 Karten.

  \item
  Die oberste Karte des gemischten Stapels wird offen in die Tischmitte gelegt.

  \item
  Die übrigen 30 Karten werden verdeckt so ausgeteilt,
  dass jede/r Spieler*in gleich viele Karten erhält.
  Vor Spielbeginn dürfen die Karten nicht angeschaut werden.
  Überschüssige Karten werden nicht benötigt und werden daher beiseite gelegt.

  \item
  Wenn alle bereit sind, nehmen alle Spieler*innen ihre Stapel so in die Hand,
  dass immer jeweils nur die oberste Karte für die/den jeweilige/n Spieler*in
  offen sichtbar ist.

  \item
  Alle Spieler*innen sind immer gleichzeitig an der Reihe.
  Auf jeder Karte befinden sich sechs Symbole.
  Für jede/n Spieler*in gibt es immer genau ein Symbol,
  das sich sowohl auf der obersten Karte auf der eigenen Hand
  als auch auf der obersten Karte des Stapels in der Tischmitte befindet.
  Wenn der/die Spieler*in den Namen "`seines/ihres"' Symbols laut ansagt,
  darf er/sie gleichzeitig die Karte offen auf den Stapel in der Tischmitte legen.

  \item
  Wenn eine Karte auf den Stapel in der Tischmitte gelegt wird,
  so ändert sich das jeweilige gesuchte Symbol für alle Spieler*innen,
  da immer die momentan oberste Karte des Stapels in der Tischmitte herangezogen wird.

  \item
  Falls zwei Spieler*innen gleichzeitig Karten ablegen, so hat der/die Spieler*in Vorrang,
  dessen/deren Karte zuerst auf dem Stapel in der Tischmitte war.

  \pagebreak

  \item
  Wer eine Karte ablegt, indem er/sie ein falsches Symbol ansagt,
  und nicht von selbst die Karte wieder aufnimmt,
  bevor ein/e andere/r Spieler*in eine andere Karte darauf ablegt oder auf den Fehler hinweist,
  hat automatisch verloren und scheidet aus dem Spiel aus.
  Dies gilt auch, wenn es sich um die letzte Karte auf der Hand handelt.

  \item
  Das Spiel endet, sobald alle Spieler*innen ihre Karten abgelegt haben.
  Wenn ein/e Spieler*in keine Karten mehr hat, so spielen die anderen Spieler*innen weiter.

  \item
  Gewonnen hat, wer als Erste/r ihre/seine Karten abgelegt hat.
\end{enumerate}

\section*{Die Symbole}

Bei der Ansage der Symbole müssen nicht genau die hier angegebenen Namen verwendet werden.
Jeder Name ist erlaubt, bei dem es bei gesundem Menschenverstand klar ist,
um welches Symbol es sich handelt.
Dazu muss
\begin{itemize}
  \item
  der Name sich um eine weit verbreitete Bezeichnung für das Symbol handeln
  (also Substantive -- nur "`rot"' für das Herz ist nicht zulässig) und

  \item
  der Name unter allen Symbolen eindeutig sein
  ("`Uni"' statt "`Uni Stuttgart"' ist zum Beispiel erlaubt,
  da es sonst keine Universität unter den Symbolen gibt).
\end{itemize}

\pagebreak

{
  \setlength{\parindent}{0mm}
  \tikzset{inner sep=0mm}

  \vspace*{\fill}

  \begin{center}
    \vspace*{\fill}

    \begin{tikzpicture}
      \node at (0,6) {\symbolImage{0.86}{0.32}};
      \node at (3.5,6) {\symbolName};
      \node at (2.5,4) {\symbolImage{0.66}{0.84}};
      \node at (4,4) {\symbolName};
      \node at (0,2.5) {\symbolImage{0.91}{0.35}};
      \node at (0,1.5) {\symbolName};
      \node at (4,1) {\symbolImage{0.84}{0.87}};
      \node at (2,1) {\symbolName};
    \end{tikzpicture}

    \vspace*{\fill}
  \end{center}

  \pagebreak

  \begin{center}
    \vspace*{\fill}

    \begin{tikzpicture}
      \node at (0,5) {\symbolImage{0.86}{0.42}};
      \node at (0,6) {\symbolName};
      \node at (4,4) {\symbolImage{0.66}{0.71}};
      \node at (2,3.5) {\symbolName};
      \node at (0,2) {\symbolImage{0.96}{0.50}};
      \node at (0,0.7) {\symbolName};
      \node at (4,0) {\symbolImage{0.58}{0.54}};
      \node at (2,-0.3) {\symbolName};
    \end{tikzpicture}

    \vspace*{\fill}
  \end{center}

  \pagebreak

  \begin{center}
    \vspace*{\fill}

    \begin{tikzpicture}
      \node at (1,6) {\symbolImage{0.53}{0.86}};
      \node at (-0.8,6) {\symbolName};
      \node at (4,5.5) {\symbolImage{0.58}{0.52}};
      \node at (4,6.8) {\symbolName};
      \node at (0,2) {\symbolImage{0.61}{0.60}};
      \node at (0,3.5) {\symbolName};
      \node at (3,2) {\symbolImage{0.35}{0.77}};
      \node at (4.6,2) {\symbolName};
    \end{tikzpicture}

    \vspace*{\fill}
  \end{center}

  \pagebreak

  \begin{center}
    \vspace*{\fill}

    \begin{tikzpicture}
      \node at (4,6) {\symbolImage{0.79}{0.37}};
      \node at (2,6) {\symbolName};
      \node at (0,4) {\symbolImage{0.73}{0.68}};
      \node at (2.3,4.5) {\symbolName};
      \node at (4,2.5) {\symbolImage{0.74}{0.83}};
      \node at (4,0.9) {\symbolName};
      \node at (0,1) {\symbolImage{0.73}{0.65}};
      \node at (1.3,0.2) {\symbolName};
    \end{tikzpicture}

    \vspace*{\fill}
  \end{center}

  \pagebreak

  \begin{center}
    \vspace*{\fill}

    \begin{tikzpicture}
      \node at (0,5) {\symbolImage{0.79}{0.41}};
      \node at (0,6.2) {\symbolName};
      \node at (3.2,5) {\symbolImage{0.77}{0.77}};
      \node at (3.2,6.7) {\symbolName};
      \node at (0,2) {\symbolImage{0.75}{0.73}};
      \node at (0,0.3) {\symbolName};
      \node at (3.2,2) {\symbolImage{0.58}{0.69}};
      \node at (3.2,0.3) {\symbolName};
    \end{tikzpicture}

    \vspace*{\fill}
  \end{center}

  \pagebreak

  \begin{center}
    \vspace*{\fill}

    \begin{tikzpicture}
      \node at (0,6) {\symbolImage{0.55}{0.71}};
      \node at (1.8,6.5) {\symbolName};
      \node at (4,4.5) {\symbolImage{0.93}{0.56}};
      \node at (1.3,4) {\symbolName};
      \node at (0,2.5) {\symbolImage{0.88}{0.38}};
      \node at (2.2,2.5) {\symbolName};
      \node at (4,1) {\symbolImage{0.84}{0.52}};
      \node at (1.9,1) {\symbolName};
    \end{tikzpicture}

    \vspace*{\fill}
  \end{center}

  \pagebreak

  \begin{center}
    \vspace*{\fill}

    \begin{tikzpicture}
      \node at (0.9,5) {\symbolImage{0.80}{0.82}};
      \node at (0.9,6.7) {\symbolName};
      \node at (4.3,5.4) {\symbolImage{0.89}{0.57}};
      \node at (4.3,4) {\symbolName};
      \node at (0,2) {\symbolImage{0.57}{0.83}};
      \node at (1.8,2.5) {\symbolName};
      \node at (4.7,2) {\symbolImage{0.95}{0.70}};
      \node at (2.8,1.5) {\symbolName};
    \end{tikzpicture}

    \vspace*{\fill}
  \end{center}

  \pagebreak

  \begin{center}
    \vspace*{\fill}

    \begin{tikzpicture}
      \node at (0,6) {\symbolImage{0.83}{0.84}};
      \node at (2.9,6) {\symbolName};
      \node at (3.7,3.75) {\symbolImage{0.81}{0.81}};
      \node at (1.1,3.75) {\symbolName};
      \node at (0,1.5) {\symbolImage{0.65}{0.73}};
      \node at (2,1.5) {\symbolName};
    \end{tikzpicture}

    \vspace*{\fill}
  \end{center}
}

\section*{Spielvarianten}

Alle Spielvarianten von "`Dobble"' und "`Spot It!"' können auch mit \theda{} gespielt werden.
Für Informationen, welche Varianten es gibt und wie man diese spielt, wird auf die Anleitungen
von "`Dobble"' und "`Spot It!"' verwiesen.

\section*{Impressum}

\begin{itemize}
  \item
  Spielidee: Jacques Cottereau, Denis Blanchot

  \item
  Bilder, Texte, Herstellung: Julian Valentin
\end{itemize}

\end{document}
