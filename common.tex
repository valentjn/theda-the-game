% PDF links
\usepackage{hyperref}

% amsmath with improvements
\usepackage{mathtools}

% use GoMono as typewriter font
\usepackage[
  % otherwise bold is not available ("undefined font shape" warnings)
  type1,
  % scale to match main font size
  scale=0.88,
]{GoMono}

% need T1 font encoding for Charter,
% otherwise there will be "undefined font shape" warnings
\usepackage[T1]{fontenc}

% silence warnings
\usepackage{silence}

% use Bitstream Charter as main font,
% suppress warning due to https://github.com/latex3/latex2e/issues/299
\WarningFilter{latex}{Font shape declaration has incorrect series value `mc'.}
\usepackage[bitstream-charter]{mathdesign}

% specify fonts to use
\usepackage{fontspec}

% symbols and names of CC licenses
\usepackage[type=CC,modifier=by-sa,version=4.0]{doclicense}

% control enumerate and itemize
\usepackage{enumitem}

% micro-typographic adjustments
\usepackage{microtype}

% colors
\usepackage{xcolor}

% graphics
\usepackage{graphicx}

% drawings
\usepackage{tikz}

% contours
\usepackage[outline]{contour}

% output stuff after the current page
\usepackage{afterpage}

\usetikzlibrary{
  % text along circle
  decorations.text,
}

% set thickness of contours
\contourlength{0.01em}

% set Charter as main font (otherwise, sz ligature will be typeset as "SS")
\setmainfont{XCharter}

% don't use sans-serif font for headings
\setkomafont{disposition}{\normalcolor\bfseries}

% define colors (mix between MATLAB and matplotlib colors)
\definecolor{C0}{rgb}{0.000,0.447,0.741}
\definecolor{C1}{rgb}{0.850,0.325,0.098}

% set up hyperref
\hypersetup{
  pdfauthor={Julian Valentin},
  pdfcreator={LaTeX, KOMA-Script, hyperref},
  % underline links instead of putting a framed box around them
  pdfborderstyle={/S/U/W 1},
  % set link colors
  citebordercolor=C1,
  filebordercolor=C1,
  linkbordercolor=C1,
  menubordercolor=C1,
  runbordercolor=C1,
  urlbordercolor=C0,
  % prepend bookmarks with section number
  bookmarksnumbered,
}

% number sets
\newcommand*{\nat}{\mathbb{N}}
\newcommand*{\integer}{\mathbb{Z}}

% notation
\newcommand*{\field}[1]{\integer/n\integer}
\newcommand*{\powerset}[1]{\mathcal{P}(#1)}

% binary relations
\newcommand*{\ceq}{\coloneqq}
\newcommand*{\dcup}{\mathbin{\dot{\cup}}}

% commands
\newcommand*{\term}[1]{\emph{#1}}
\newcommand*{\theda}[1]{%
  \texorpdfstring{$\vartheta$}{ϑ}~--~\textsc{Das~Spiel}%
  \if\relax\detokenize{#1}\relax%
  \else%
    ~--~#1%
  \fi%
}

% macros for including pages
\newcommand*{\pageSize}{8.1}
\newcommand*{\truncationAngle}{30}
\pgfmathsetmacro{\pageRadius}{\pageSize/sqrt(2)}
\pgfmathsetmacro{\truncationPointTopX}{\pageRadius*cos(180-\truncationAngle)}
\pgfmathsetmacro{\truncationPointTopY}{\pageRadius*sin(180-\truncationAngle)}

% include a single right page
\newcommand*{\rightPage}[4]{
  \if\relax\detokenize{#4}\relax
  \else
    \node at (#1,#2) {\includegraphics[page=#4]{#3}};
    \draw (#1-\pageSize/2,#2+\pageSize/2)
        -- (#1+\pageSize/2,#2+\pageSize/2)
        -- (#1+\pageSize/2,#2-\pageSize/2)
        -- (#1-\pageSize/2,#2-\pageSize/2)
        arc[start angle=225,end angle=180+\truncationAngle,radius=\pageRadius]
        -- (#1+\truncationPointTopX,#2+\truncationPointTopY)
        arc[start angle=180-\truncationAngle,end angle=135,radius=\pageRadius];
  \fi
}

\newcommand*{\leftPage}[4]{
  \if\relax\detokenize{#4}\relax
  \else
    \node at (#1,#2) {\includegraphics[page=#4]{#3}};
    \draw[white] (#1+\pageSize/2,#2+\pageSize/2)
        -- (#1-\pageSize/2,#2+\pageSize/2)
        -- (#1-\pageSize/2,#2-\pageSize/2)
        -- (#1+\pageSize/2,#2-\pageSize/2)
        arc[start angle=-45,end angle=-\truncationAngle,radius=\pageRadius]
        -- (#1-\truncationPointTopX,#2+\truncationPointTopY)
        arc[start angle=\truncationAngle,end angle=45,radius=\pageRadius];
  \fi
}

% place six right pages on one sheet of paper
\newcommand*{\nupRightPage}[7]{%
  \vspace*{\fill}%
  %
  \begin{center}%
    \begin{tikzpicture}
      \rightPage{0}{18}{#1}{#2}
      \rightPage{10}{18}{#1}{#3}
      \rightPage{0}{9}{#1}{#4}
      \rightPage{10}{9}{#1}{#5}
      \rightPage{0}{0}{#1}{#6}
      \rightPage{10}{0}{#1}{#7}
    \end{tikzpicture}%
  \end{center}%
  %
  \vspace*{\fill}%
  \pagebreak%
  \par%
}

% place six left pages on one sheet of paper
\newcommand*{\nupLeftPage}[7]{%
  \vspace*{\fill}%
  %
  \begin{center}%
    \begin{tikzpicture}
      \leftPage{0}{18}{#1}{#2}
      \leftPage{10}{18}{#1}{#3}
      \leftPage{0}{9}{#1}{#4}
      \leftPage{10}{9}{#1}{#5}
      \leftPage{0}{0}{#1}{#6}
      \leftPage{10}{0}{#1}{#7}
    \end{tikzpicture}%
  \end{center}%
  %
  \vspace*{\fill}%
  \pagebreak%
  \par%
}
