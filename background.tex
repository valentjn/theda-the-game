\documentclass[paper=81mm:81mm,fontsize=10.7pt]{scrartcl}
\usepackage[ngerman]{babel}

% PDF links
\usepackage{hyperref}

% amsmath with improvements
\usepackage{mathtools}

% use GoMono as typewriter font
\usepackage[
  % otherwise bold is not available ("undefined font shape" warnings)
  type1,
  % scale to match main font size
  scale=0.88,
]{GoMono}

% need T1 font encoding for Charter,
% otherwise there will be "undefined font shape" warnings
\usepackage[T1]{fontenc}

% silence warnings
\usepackage{silence}

% use Bitstream Charter as main font,
% suppress warning due to https://github.com/latex3/latex2e/issues/299
\WarningFilter{latex}{Font shape declaration has incorrect series value `mc'.}
\usepackage[bitstream-charter]{mathdesign}

% specify fonts to use
\usepackage{fontspec}

% symbols and names of CC licenses
\usepackage[type=CC,modifier=by-sa,version=4.0]{doclicense}

% control enumerate and itemize
\usepackage{enumitem}

% micro-typographic adjustments
\usepackage{microtype}

% colors
\usepackage{xcolor}

% graphics
\usepackage{graphicx}

% drawings
\usepackage{tikz}

% contours
\usepackage[outline]{contour}

% output stuff after the current page
\usepackage{afterpage}

\usetikzlibrary{
  % text along circle
  decorations.text,
}

% set thickness of contours
\contourlength{0.01em}

% set Charter as main font (otherwise, sz ligature will be typeset as "SS")
\setmainfont{XCharter}

% don't use sans-serif font for headings
\setkomafont{disposition}{\normalcolor\bfseries}

% define colors (mix between MATLAB and matplotlib colors)
\definecolor{C0}{rgb}{0.000,0.447,0.741}
\definecolor{C1}{rgb}{0.850,0.325,0.098}

% set up hyperref
\hypersetup{
  pdfauthor={Julian Valentin},
  pdfcreator={LaTeX, KOMA-Script, hyperref},
  % underline links instead of putting a framed box around them
  pdfborderstyle={/S/U/W 1},
  % set link colors
  citebordercolor=C1,
  filebordercolor=C1,
  linkbordercolor=C1,
  menubordercolor=C1,
  runbordercolor=C1,
  urlbordercolor=C0,
  % prepend bookmarks with section number
  bookmarksnumbered,
}

% number sets
\newcommand*{\nat}{\mathbb{N}}
\newcommand*{\integer}{\mathbb{Z}}

% notation
\newcommand*{\field}[1]{\integer/n\integer}
\newcommand*{\powerset}[1]{\mathcal{P}(#1)}

% binary relations
\newcommand*{\ceq}{\coloneqq}
\newcommand*{\dcup}{\mathbin{\dot{\cup}}}

% commands
\newcommand*{\term}[1]{\emph{#1}}
\newcommand*{\theda}[1]{%
  \texorpdfstring{$\vartheta$}{ϑ}~--~\textsc{Das~Spiel}%
  \if\relax\detokenize{#1}\relax%
  \else%
    ~--~#1%
  \fi%
}

% macros for including pages
\newcommand*{\pageSize}{8.1}
\newcommand*{\truncationAngle}{30}
\pgfmathsetmacro{\pageRadius}{\pageSize/sqrt(2)}
\pgfmathsetmacro{\truncationPointTopX}{\pageRadius*cos(180-\truncationAngle)}
\pgfmathsetmacro{\truncationPointTopY}{\pageRadius*sin(180-\truncationAngle)}

% include a single right page
\newcommand*{\rightPage}[4]{
  \if\relax\detokenize{#4}\relax
  \else
    \node at (#1,#2) {\includegraphics[page=#4]{#3}};
    \draw (#1-\pageSize/2,#2+\pageSize/2)
        -- (#1+\pageSize/2,#2+\pageSize/2)
        -- (#1+\pageSize/2,#2-\pageSize/2)
        -- (#1-\pageSize/2,#2-\pageSize/2)
        arc[start angle=225,end angle=180+\truncationAngle,radius=\pageRadius]
        -- (#1+\truncationPointTopX,#2+\truncationPointTopY)
        arc[start angle=180-\truncationAngle,end angle=135,radius=\pageRadius];
  \fi
}

\newcommand*{\leftPage}[4]{
  \if\relax\detokenize{#4}\relax
  \else
    \node at (#1,#2) {\includegraphics[page=#4]{#3}};
    \draw[white] (#1+\pageSize/2,#2+\pageSize/2)
        -- (#1-\pageSize/2,#2+\pageSize/2)
        -- (#1-\pageSize/2,#2-\pageSize/2)
        -- (#1+\pageSize/2,#2-\pageSize/2)
        arc[start angle=-45,end angle=-\truncationAngle,radius=\pageRadius]
        -- (#1-\truncationPointTopX,#2+\truncationPointTopY)
        arc[start angle=\truncationAngle,end angle=45,radius=\pageRadius];
  \fi
}

% place six right pages on one sheet of paper
\newcommand*{\nupRightPage}[7]{%
  \vspace*{\fill}%
  %
  \begin{center}%
    \begin{tikzpicture}
      \rightPage{0}{18}{#1}{#2}
      \rightPage{10}{18}{#1}{#3}
      \rightPage{0}{9}{#1}{#4}
      \rightPage{10}{9}{#1}{#5}
      \rightPage{0}{0}{#1}{#6}
      \rightPage{10}{0}{#1}{#7}
    \end{tikzpicture}%
  \end{center}%
  %
  \vspace*{\fill}%
  \pagebreak%
  \par%
}

% place six left pages on one sheet of paper
\newcommand*{\nupLeftPage}[7]{%
  \vspace*{\fill}%
  %
  \begin{center}%
    \begin{tikzpicture}
      \leftPage{0}{18}{#1}{#2}
      \leftPage{10}{18}{#1}{#3}
      \leftPage{0}{9}{#1}{#4}
      \leftPage{10}{9}{#1}{#5}
      \leftPage{0}{0}{#1}{#6}
      \leftPage{10}{0}{#1}{#7}
    \end{tikzpicture}%
  \end{center}%
  %
  \vspace*{\fill}%
  \pagebreak%
  \par%
}


% set KOMA-Script text area
\areaset{77mm}{77mm}

% set PDF title
\hypersetup{pdftitle={\theda{Der Hintergrund}}}

\begin{document}

\vspace*{\fill}

\noindent\hspace*{9mm}%
\begin{tikzpicture}
  \node at (0,0) {%
    \fontsize{200pt}{240pt}\selectfont%
    \contour{black}{\textcolor{white}{$\vartheta$}}%
  };
  \node[align=center,anchor=north,inner sep=0mm] at (3,0.5) {%
    \LARGE\textsc{Das}\\[0.5em]%
    \LARGE\textsc{Spiel}%
  };
  \node[align=center,anchor=north,inner sep=0mm] at (3,-1.3) {%
    \LARGE\textsc{Der}\\[0.5em]%
    \LARGE\textsc{Hinter-}\\[0.5em]%
    \LARGE\textsc{\vphantom{H}grund}%
  };
  \draw (2.6,-0.95) -- (3.4,-0.95);
\end{tikzpicture}

\vspace*{\fill}

{%
  \noindent\small%
  Bilder/Texte: Copyright \textcopyright{} 2021 Julian Valentin,\\
  lizenziert unter \doclicenseIcon{} CC BY-SA 4.0%
}

\pagebreak

\section*{Einleitung}

\theda{} ist eine kleinere Version von "`Dobble"' und "`Spot It!"' und hat den gleichen
mathematischen Hintergrund.
All diese Spiele bestehen aus Karten mit verschiedenen Symbolen darauf,
die folgende Voraussetzungen erfüllen:

\begin{enumerate}[label=G\arabic*.]
  \item
  Jede Karte hat mit jeder anderen Karte genau ein Symbol gemein.

  \item
  Auf jeder Karte befinden sich gleich viele Symbole.

  \item
  Auf jeder Karte befindet sich jedes Symbol höchstens ein Mal.

  \item
  Kein Symbol befindet sich auf jeder Karte.
\end{enumerate}

Die Originalversionen haben 55 Karten mit insgesamt 57 Symbolen,
wobei sich auf jeder Karte acht Symbole befinden.
Auf den ersten Blick erscheint es erstaunlich,
dass alle Voraussetzungen von so vielen Karten und Symbolen erfüllt werden können.
Wie kann man ein solch großes Kartendeck konstruieren?

Dazu gibt es verschiedene Ansätze.
Dieser Artikel stellt einen (bereits bekannten) Ansatz vor, mit dem für beliebige Primzahlen $n$
ein Kartendeck mit $n^2 + n + 1$ Karten und Symbolen und $n + 1$ Symbolen pro Karte konstruiert
werden kann.
Für \theda{} wurde $n = 5$ gewählt.
Die Originalversionen basieren auf demselben Ansatz, wenn man $n = 7$ wählt und
zwei der möglichen $n^2 + n + 1 = 57$ Karten weglässt.

\pagebreak

Die Idee für diesen Artikel stammt von
\href{%
  https://www.petercollingridge.co.uk/blog/mathematics-toys-and-games/dobble/%
}{%
  \texttt{%
    https:\\%
    //www.petercollingridge.co.uk/blog/\\%
    mathematics-toys-and-games/dobble/%
  }%
}.

\section*{Konstruktion eines Spiels}

Zunächst definieren wir die Menge $S$ der Symbole als $S \ceq S^{(<\infty)} \dcup S^{(\infty)}$ mit
den \term{endlichen Symbolen} $S^{(<\infty)} \ceq \{0, \dotsc, n - 1\}^2$ und
den \term{Fernsymbolen} $S^{(\infty)} \ceq \{0, \dotsc, n\}$.

Anschließend definieren wir die Menge $C$ der Karten wie folgt:
\begin{align*}
  C &\ceq \{c^\mathrm{(d)}_{a,b} \mid a, b = 0, \dotsc, n - 1\}\\
  &{} \qquad\dcup \{c^\mathrm{(r)}_i \mid i = 0, \dotsc, n - 1\} \dcup \{c^{(\infty)}\},
\end{align*}
wobei für $a, b, i = 0, \dotsc, n - 1$
\begin{align*}
  c^\mathrm{(d)}_{a,b}
  &\ceq
  \{
    (i, (a + bi) \bmod n) \mid i \!=\! 0, \scalebox{0.74}[1]{$\dotsc$}, n \!-\! 1
  \}
  \!\dcup\! \{b\},\\
  c^\mathrm{(r)}_i &\ceq \{(i, j) \mid j = 0, \dotsc, n - 1\} \dcup \{n\},\\
  c^{(\infty)} &\ceq S^{(\infty)} = \{0, \dotsc, n\}
\end{align*}
jeweils die \term{Diagonalkarten,} die \term{Zeilenkarten} und die \term{Fernkarte} definieren.

\afterpage{%
  \vspace*{\fill}%
  %
  \begin{center}%
    \small%
    \begin{tikzpicture}[scale=1.2]
      \clip (-1.4,-2.5) rectangle (4.6,3);

      \node[circle,draw,inner sep=0mm] (point00) at (0,2) {$(0, 0)$};
      \node[circle,draw,inner sep=0mm] (point01) at (1,2) {$(0, 1)$};
      \node[circle,draw,inner sep=0mm] (point02) at (2,2) {$(0, 2)$};
      \node[circle,draw,inner sep=0mm] (point10) at (0,1) {$(1, 0)$};
      \node[circle,draw,inner sep=0mm] (point11) at (1,1) {$(1, 1)$};
      \node[circle,draw,inner sep=0mm] (point12) at (2,1) {$(1, 2)$};
      \node[circle,draw,inner sep=0mm] (point20) at (0,0) {$(2, 0)$};
      \node[circle,draw,inner sep=0mm] (point21) at (1,0) {$(2, 1)$};
      \node[circle,draw,inner sep=0mm] (point22) at (2,0) {$(2, 2)$};

      \node[circle,draw] (point0) at (-1,-2) {$1$};
      \node[circle,draw] (point1) at (1,-2) {$0$};
      \node[circle,draw] (point2) at (3,-2) {$2$};
      \node[circle,draw] (point3) at (4,1) {$3$};

      \draw[dash pattern=on 3pt off 2pt on 1pt off 2pt]
          (point00) -- (point11) -- (point22) to[out=-45,in=30] (point0);
      \draw[dash pattern=on 3pt off 2pt on 1pt off 2pt]
          (point01) -- (point12) to[out=-45,in=-45,looseness=2] (point20)
          to[out=135,in=135] (point0);
      \draw[dash pattern=on 3pt off 2pt on 1pt off 2pt]
          (point02) to[out=135,in=135,looseness=2] (point10) -- (point21)
          to[out=-45,in=45,looseness=1.5] (point0);

      \draw[dashed] (point00) -- (point10) -- (point20) to[out=-90,in=135] (point1);
      \draw[dashed] (point01) -- (point11) -- (point21) -- (point1);
      \draw[dashed] (point02) -- (point12) -- (point22) to[out=-90,in=45] (point1);

      \draw[dash pattern=on 3pt off 2pt on 1pt off 2pt on 1pt off 2pt]
          (point00) to[out=45,in=45,looseness=2] (point12) -- (point21)
          to[out=-135,in=135,looseness=1.5] (point2);
      \draw[dash pattern=on 3pt off 2pt on 1pt off 2pt on 1pt off 2pt]
          (point01) -- (point10) to[out=-135,in=-135,looseness=2] (point22)
          to[out=45,in=45] (point2);
      \draw[dash pattern=on 3pt off 2pt on 1pt off 2pt on 1pt off 2pt]
          (point02) -- (point11) -- (point20) to[out=-135,in=150] (point2);

      \draw (point00) -- (point01) -- (point02) to[out=0,in=135] (point3);
      \draw (point10) -- (point11) -- (point12) -- (point3);
      \draw (point20) -- (point21) -- (point22) to[out=0,in=-135] (point3);

      \draw[very thick] (point0) -- (point1) -- (point2) to[out=0,in=-45] (point3);
    \end{tikzpicture}%
  \end{center}%
  %
  \vspace*{\fill}%
  %
  \pagebreak%
}

Man erhält so ein Spiel mit $|S| = n^2 + n + 1 = |C|$ Symbolen und Karten.
Man kann sich leicht davon überzeugen, dass keine Karte und
kein Symbol auf einer Karte mehrfach vorkommen.
Bei der Herstellung der tatsächlichen Karten kann man natürlich alle Symbole umbenennen
und durch grafische Symbole ersetzen, solange diese paarweise verschieden sind.

Die Namen "`endliche Symbole"', "`Fernsymbole"' und "`Fernkarte"'
sind übrigens an den Zusammenhang zu endlichen projektiven Ebenen angelehnt,
der am Ende erläutert wird.

\section*{Beweis der Zulässigkeit}

Im Folgenden vollziehen wir nach, dass das oben konstruierte Spiel $(S, C)$ zulässig ist,
dass also die Voraussetzungen G1 bis G4 für $S$ und $C$ gelten.

Die Voraussetzungen G2 bis G4 lassen sich leicht prüfen:
G2 gilt, da $|c^\mathrm{(d)}_{a,b}| = |\smash{c^\mathrm{(r)}_i}| = |c^{(\infty)}| = n + 1$
für alle $a, b, i = 0, \dotsc, n - 1$.
Außerdem gilt G3, da die Karten jeweils Mengen sind, die kein Element mehrfach enthalten können.
Schließlich gilt G4, da zum Beispiel das Fernsymbol $n$ nur auf den Zeilenkarten und
der Fernkarte vorhanden ist, nicht aber auf den Diagonalkarten.

Um G1 zu zeigen, betrachtet man die verschiedenen Arten der Karten.
Klar ist, dass die Fernkarte mit allen anderen Karten genau ein Symbol gemein hat,
da jede dieser Karten ein Fernsymbol enthält (und die Fernkarte enthält alle Fernsymbole).
Ebenso einfach sieht man, dass jede Zeilenkarte mit allen anderen Zeilenkarten und
allen Diagonalkarten genau ein Symbol gemein hat.
Zwei verschiedene Zeilenkarten haben nämlich genau das Symbol $n$ gemein, und\vspace{-0.1em}
die Zeilenkarte $c^\mathrm{(r)}_i$ hat mit der Diagonalkarte $c^\mathrm{(d)}_{a,b}$
genau das Symbol $(i, (a + bi) \bmod n)$ gemein.

Die eigentliche Schwierigkeit ist also der Beweis,
dass zwei verschiedene Diagonalkarten jeweils genau ein Symbol gemein haben.
Dazu muss man etwas tiefer in die algebraische Trickkiste greifen.
Zunächst bemerken wir, dass es zwei Fälle gibt,
wenn man zwei Diagonalkarten $c^\mathrm{(d)}_{a',b'}$ und $c^\mathrm{(d)}_{a'',b''}$ vor sich hat:
\vspace{-0.1em}$b' = b''$ oder $b' \not= b''$.

Im ersten Fall $b' = b''$ gilt\vspace{-0.1em}
$b' \in c^\mathrm{(d)}_{a',b'} \cap c^\mathrm{(d)}_{a'',b''}$.
Es verbleibt zu zeigen,
dass kein Paar aus $S^{(<\infty)}$ in beiden Karten enthalten ist.
Wenn wir das Gegenteil annehmen, dann gibt es ein $i$ mit
$a' + b'i \equiv_n a'' + b''i$, wobei "`$\equiv_n$"' die Gleichheit modulo $n$ anzeigt.
Subtraktion von $b'i = b''i$ auf beiden Seiten führt zu $a' \equiv_n a''$,
somit müssen die beiden Karten gleich sein.
Im Umkehrschluss kann bei zwei verschiedenen Diagonalkarten mit $b' = b''$ nur $b'$
in beiden enthalten sein.

Im zweiten Fall $b' \not= b''$ ist kein Fernsymbol in beiden Karten gleichzeitig enthalten.
Daher ist zu zeigen,
dass genau ein Paar aus $S^{(<\infty)}$ in beiden Karten enthalten ist.
Dies ist äquivalent dazu, dass es genau ein $i \in \{0, \dotsc, n - 1\}$ gibt mit
\begin{alignat*}{2}
  && a' + b'i &\equiv_n a'' + b''i.\\
  \intertext{%
    Nun benutzen wir, dass $\field{n}$ für primes $n$ ein Körper ist.
    Da $b'' - b' \not\equiv_n 0$ in einem Körper ein multiplikativ Inverses hat
    (das lässt sich mit dem Lemma von Bézout zeigen), erhalten wir%
  }
  \iff\quad&& (b'' - b') i &\equiv_n a' - a''\\
  \iff\quad&& i &\equiv_n (b'' - b')^{-1} (a' - a'').
\end{alignat*}
Wenn man also $i = ((b'' - b')^{-1} (a' - a'')) \bmod n$ wählt,
dann ist $(i, (a' + b'i) \bmod n) \in c^\mathrm{(d)}_{a',b'} \cap c^\mathrm{(d)}_{a'',b''}$.
Da $i$ eindeutig bestimmt ist, gibt es kein anderes Symbol in der Schnittmenge.

Wir haben damit für alle Fälle gezeigt, dass G1 gilt;
somit ist $(S, C)$ ein zulässiges Spiel.

\section*{Ausblick: Zusammenhang zu projektiven Ebenen und lateinischen Quadraten}

Abschließend wollen wir noch einige enge Zusammenhänge zu projektiven Ebenen und
lateinischen Quadraten betrachten, auf denen obige Konstruktion beruht.

\subsection*{Endliche projektive Ebenen}

Eine \term{endliche projektive Ebene (EPE)} der Ordnung $n \in \nat \ceq \{1, 2, 3, \dotsc\}$
ist ein Paar $(P, L)$
einer endlichen Menge $P$ \term{(Punkte)} und
einer Teilmenge $L \subset \powerset{P}$ der Potenzmenge von $P$ \term{(Geraden),}
sodass gilt:
\begin{enumerate}[label=P\arabic*.]
  \item
  Es gibt $n^2 + n + 1$ Punkte und Geraden:
  Es gilt $|P| = n^2 + n + 1 = |L|$.

  \item
  Durch zwei verschiedene Punkte verläuft genau eine Gerade:
  Für alle $p', p'' \in P$ mit $p' \not= p''$ ist $|\{\ell \in L \mid p', p'' \in \ell\}| = 1$.

  \item
  Zwei verschiedene Geraden schneiden sich in genau einem Punkt:
  Für alle $\ell', \ell'' \in L$ mit $\ell' \not= \ell''$ gilt $|\ell' \cap \ell''| = 1$.

  \item
  Jeder Punkt liegt auf $n + 1$ Geraden:
  Für alle $p \in P$ ist $|\{\ell \in L \mid p \in \ell\}| = n + 1$.

  \item
  Jede Gerade verläuft durch $n + 1$ Punkte:
  Für alle $\ell \in L$ ist $|\ell| = n + 1$.
\end{enumerate}

Im Gegensatz zur herkömmlichen affinen Geometrie schneiden sich in der projektiven Geometrie
auch parallele Geraden; der Schnittpunkt wird als ein \term{Fernpunkt} bezeichnet.
Geraden, die nur \term{Fernpunkte} enthalten, heißen \term{Ferngeraden.}

Aus jeder EPE $(P, L)$ lässt sich ein zulässiges Spiel $(S, C)$
konstruieren, indem man $S \ceq P$ und $C \ceq L$ wählt.
Umgekehrt lässt sich obige Konstruktion des Spiels verwenden, um eine endliche projektive Ebene
der Ordnung $n$ zu konstruieren.
Die Spiele-Konstruktion ist erweiterbar für Primzahlpotenzen $n$.
In diesem Fall wird die Formulierung komplexer, da dann $\field{n}$ kein Körper ist
(es gibt aber andere Körper).

Die Existenz von EPEs der Ordnung $n$ ist nur für folgende Fälle bekannt:

\begin{itemize}
  \item
  Eine EPE existiert, falls $n$ eine Primzahlpotenz ist (z.\,B. $n = 5$ oder $n = 9 = 3^2$).

  \item
  Eine EPE existiert nicht, falls [$n \equiv_4 1$ oder $n \equiv_4 2$] und
  $n$ nicht die Summe zweier Quadratzahlen ist (z.\,B. $n = 6$).

  \item
  Eine EPE existiert nicht, falls $n = 10$ (Beweis durch Brute-Force).
\end{itemize}
Im Allgemeinen ist die Existenz von EPEs ein großes ungelöstes Problem der Kombinatorik.

\subsection*{Lateinische Quadrate}

Ein \emph{lateinisches Quadrat} der Ordnung $n \in \nat$ ist eine $(n \times n)$-Matrix
mit Einträgen aus einer Menge mit $n$ Elementen,
sodass jedes Element der Menge in jeder Zeile und in jeder Spalte genau ein Mal vorkommt.

Zwei lateinische Quadrate der Ordnung $n$ sind \emph{orthogonal,}
falls die $(n \times n)$-Matrix, die man erhält,
wenn man die Einträge der beiden Quadrate zu je einem Paar kombiniert,
kein Paar mehrfach enthält.

Eine \emph{vollständige Menge von paarweise orthogonalen lateinischen Quadraten (VMPOLQ)}
der Ordnung $n$ ist eine Menge von $n - 1$ lateinischen Quadraten der Ordnung $n$,
die paarweise orthogonal sind.

Man kann beweisen, dass eine VMPOLQ äquivalent zu einer sog. \term{endlichen affinen Ebene} ist,
die wiederum eindeutig zu einer EPE erweitert werden kann.
Die Aussage, dass EPEs der Ordnung $n$ existieren, falls $n$ eine Primzahlpotenz ist,
basiert auf genau dieser Äquivalenz:
Es gibt einen Algorithmus, der eine VMPOLQ der Ordnung $n$ aus einem endlichen Körper mit
$n$ Elementen konstruiert, und ein endlicher Körper mit $n$ Elementen existiert genau dann, wenn
$n$ eine Primzahlpotenz ist.

Falls $n$ sogar eine Primzahl ist, lässt sich unsere Spielekonstruktion
(die im Prinzip ein Spezialfall des erwähnten Algorithmus ist) anwenden,
um eine VMPOLQ der Ordnung $n$ zu konstruieren.
Um das $b$-te lateinische Quadrat zu konstruieren ($b = 1, \dotsc, n - 1$),
erstellt man eine $(n \times n)$-Matrix.
Für alle $a = 0, \dotsc, n - 1$ schreibt man nun $a$ in die Einträge der Matrix,
deren nullbasierte Indizes sich als Paar in $c^\mathrm{(d)}_{a,b}$ befinden.
Man kann zeigen, dass alle Einträge der Matrix genau ein Mal beschrieben werden.
Aufwendiger ist der Beweis, dass die Matrizen tatsächlich paarweise orthogonale
lateinische Quadrate sind.

\end{document}
